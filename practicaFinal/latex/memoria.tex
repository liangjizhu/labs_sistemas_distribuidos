\documentclass[12pt,a4paper]{article}
\usepackage[utf8]{inputenc}
\usepackage[T1]{fontenc}
\usepackage[spanish]{babel}
\usepackage{graphicx}
\usepackage{listings}
\usepackage{xcolor}
\lstset{
    basicstyle=\ttfamily\small,
    keywordstyle=\color{blue}\bfseries,
    commentstyle=\color{gray}\itshape,
    stringstyle=\color{red},
    breaklines=true,
    numbers=left,
    numberstyle=\tiny,
    numbersep=5pt,
    frame=single,
    captionpos=b
    showstringspaces=false
}

\usepackage[strings]{underscore}
\begin{document}

% Portada
\begin{titlepage}
    \centering
    {\LARGE\textbf{SISTEMAS DISTRIBUIDOS}\par}
    \vspace{1cm}
    {\Large {Práctica Final: Diseño e Implementación de un Sistema \textit{peer-to-peer} (P2P)}\par}
    \vspace{1cm}
    \vfill
    \includegraphics[width=0.75\textwidth]{logo_universidad.png}
    \vfill
    \vspace{1cm}
    {\large\textbf{ID de grupo de prácticas:} 81\par}
    {\large\textbf{Nombre de los participantes:}\par}
    {\normalsize Liang Ji Zhu \\ Paolo Michael Webb\par}
    {\large\textbf{Correos electrónicos de los participantes:}\par}
    {\normalsize 100495723@alumnos.uc3m.es \\ 100495955@alumnos.uc3m.es\par}
    \vfill
    {\large Madrid, \today\par}
\end{titlepage}

% Índice automático según estructura de secciones
\clearpage
\tableofcontents
\clearpage

% Introducción
\section{Introducción}

\section{Parte 1: P2P básico}

En esta primera parte diseñamos e implementamos un sistema peer‐to‐peer simple en el que:

\begin{itemize}
  \item Cada cliente puede \textbf{registrarse} o \textbf{darse de baja} en el servidor central.
  \item Un cliente \textbf{conectado} abre un socket TCP en un puerto y se anuncia al servidor.
  \item Un cliente puede \textbf{publicar} un fichero (ruta y descripción) o \textbf{borrar} una publicación.
  \item Un cliente puede \textbf{listar} los usuarios conectados y el contenido publicado de cualquier usuario.
  \item Para \texttt{GET\_FILE}, el cliente pide al servidor central la IP y puerto remotos, abre conexión directa P2P y descarga el fichero.
\end{itemize}

\subsection{Protocolo de aplicación}

Cada mensaje entre cliente y servidor central va con terminador NUL (\texttt{\textbackslash0}) y el flujo es:

\begin{enumerate}
  \item Cliente conecta TCP al servidor.
  \item Envía la \textbf{operación} en ASCII más un \texttt{\textbackslash0}.
  \item Envía parámetros (username, filename, descripción, puerto, etc.) cada uno terminado en \texttt{\textbackslash0}.
  \item Servidor responde con un byte de código de retorno:
    \begin{description}
      \item[0:] OK.
      \item[1:] Error de usuario (no existe / ya existe / no está conectado).
      \item[2:] Error de estado (por ejemplo, usuario no conectado).
      \item[3:] Error de aplicación (contenido duplicado, no publicado, etc.).
    \end{description}
  \item Para operaciones de \texttt{LIST\_USERS} o \texttt{LIST\_CONTENT}, tras el 0 viene un entero en ASCII (\texttt{"3\textbackslash0"}) y a continuación esa cantidad de cadenas NUL‐terminated.
  \item Para \texttt{GET\_FILE}, primer byte = existencia (0 / 1), y si existe: tamaño en ASCII terminado en NUL, luego el contenido bruto.
\end{enumerate}

\subsection{Cliente Python}

La lógica del cliente está en \texttt{client.py}. Usamos la librería \texttt{socket} y un hilo secundario para el P2P.  
A continuación se muestra la cabecera de la clase y la implementación de \texttt{register} y \texttt{getfile}:

\begin{lstlisting}[language=Python,caption={Fragmento de \texttt{client.py}: register y getfile},label=lst:client1]
class client:
    @staticmethod
    def register(user):
        with socket.socket() as s:
            s.connect((client._server, client._port))
            s.sendall(b"REGISTER\0")
            s.sendall(user.encode() + b"\0")
            code = s.recv(1)[0]
            if code == 0:
                print("REGISTER OK"); client._user = user
            elif code == 1:
                print("USERNAME IN USE")
            else:
                print("REGISTER FAIL")

    @staticmethod
    def getfile(user, remote_fn, local_fn):
        # 1) Pido IP/puerto al servidor central
        with socket.socket() as si:
            si.connect((client._server, client._port))
            si.sendall(b"GET USER INFO\0")
            si.sendall(client._user.encode()+b"\0")
            si.sendall(user.encode()+b"\0")
            if si.recv(1)[0]!=0: return
            ip = client.read_string(si)
            port = int(client.read_string(si))
        # 2) Conecto P2P y descargo
        with socket.socket() as s2:
            s2.connect((ip, port))
            s2.sendall(b"GET FILE\0")
            s2.sendall(remote_fn.encode()+b"\0")
            if s2.recv(1)[0]==0:
                size = int(client.read_string(s2))
                with open(local_fn,"wb") as f:
                    remain=size
                    while remain>0:
                        chunk=s2.recv(min(1024,remain))
                        f.write(chunk); remain-=len(chunk)
                print("GET_FILE OK")
            else:
                print("GET_FILE FAIL, FILE NOT EXIST")
\end{lstlisting}

\subsection{Servidor en C}

El servidor central está en \texttt{server.c}. Usa hilos POSIX para cada conexión entrante.  
Extracto de la inicialización y la rama de \texttt{PUBLISH}:

\begin{lstlisting}[language=C,caption={Fragmento de \texttt{server.c}: arranque y PUBLISH},label=lst:server1]
int main(...) {
    signal(SIGINT, handle_sigint);
    server_socket = socket(...);
    setsockopt(server_socket, SOL_SOCKET, SO_REUSEADDR,...);
    bind(server_socket,...); listen(server_socket,10);
    printf("s> init server 0.0.0.0:%d\n", port);
    while(1) {
        int *cs = malloc(sizeof(int));
        *cs = accept(server_socket,...);
        pthread_t tid;
        pthread_create(&tid,NULL,client_handler,cs);
        pthread_detach(tid);
    }
}

void *client_handler(void *a) {
    int sock = *(int*)a; free(a);
    char op[MAX_NAME], user[MAX_NAME];
    read_string(sock,op); read_string(sock,user);
    printf("s> OPERATION %s FROM %s\n",op,user);
    // ... logica de REGISTER, CONNECT, etc. ...
    if(strcmp(op,"PUBLISH")==0) {
        char path[MAX_NAME],desc[MAX_DESC];
        read_string(sock,path); read_string(sock,desc);
        // Comprueba duplicados y anade a users[user_idx].files[]
        send(sock,&res,1,0);
    }
    close(sock); return NULL;
}
\end{lstlisting}

\subsection{Pruebas automáticas}

Para validar la Parte 1 creamos \texttt{run\_test.sh} que:

\begin{itemize}
  \item Registra dos usuarios (\texttt{paolo}, \texttt{liang}), prueba duplicados.
  \item Conecta, publica y detecta duplicados.
  \item Lista usuarios y contenidos, descarga un fichero con \texttt{GET\_FILE}.
  \item Borra contenido y comprueba fallo de \texttt{GET\_FILE}.
\end{itemize}

Al ejecutarlo obtenemos:

\begin{verbatim}
[PASS] REGISTER liang (OK)
[PASS] CONNECT liang (OK)
...
[PASS] Contenido de GET_FILE coincide
[PASS] DELETE existing content (OK)
[PASS] GET_FILE tras DELETE (FAIL)
+++ TODOS LOS TESTS SUPERADOS +++
\end{verbatim}

Este conjunto de pruebas cubre todos los casos del protocolo de la Parte 1. Así confirmamos que todas las funcionalidades son correctas, y todos los casos extremos se han tenido en cuenta.

\subsection{Makefile}

Finalmente, el \texttt{Makefile} compila cliente y servidor P2P:

\begin{lstlisting}[caption={Makefile parte 1},label=lst:make1]
CC = gcc
CFLAGS = -Wall -pthread
CLIENT_SRCS = client.py
SERVER_SRCS = server.c

all: server

server: server.c
	$(CC) $(CFLAGS) -o server server.c

clean:
	rm -f server
\end{lstlisting}

Con \texttt{make} obtenemos el binario \texttt{server} y luego ejecutamos \texttt{python3 client.py -s 127.0.0.1 -p 12345} para interactuar.

\section{Parte 2: Servicio Web}

Para añadir a la funcionalidad del P2P, en esta parte desarrollamos un servicio web local que devuelve la fecha y hora actual en el formato \texttt{DD/MM/YYYY HH:MM:SS}. Este servicio se utiliza para registrar la hora exacta en la que se realizan las operaciones del cliente.

\subsection{Diseño del servicio}

El servicio está implementado en Python usando Flask, en el archivo \texttt{datetime\_service.py}. El endpoint principal es:

\begin{itemize}
\item \texttt{GET /datetime}: devuelve una cadena con la fecha y hora actual.
\end{itemize}

\begin{lstlisting}[language=Python,caption={Fragmento de datetime\_service.py},label=lst:web1]
@app.route('/datetime')
def current_datetime():
ts = datetime.now().strftime('%d/%m/%Y %H:%M:%S')
return Response(ts, mimetype='text/plain')
\end{lstlisting}

\subsection{Integración con el cliente}

Se modificó el cliente para que en cada operación (REGISTER, CONNECT, PUBLISH, etc.):

\begin{enumerate}
\item Se realice una petición HTTP al servicio web para obtener el timestamp.
\item Se envíe este timestamp al servidor justo después del código de operación.
\end{enumerate}

Este comportamiento está encapsulado en la función \texttt{\_send\_op()}:

\begin{lstlisting}[language=Python,caption={Cliente: envío de operación con timestamp},label=lst:web2]
@staticmethod
def _send_op(s, op_str):
s.sendall(op_str.encode('utf-8') + b'\0')
ts = requests.get('http://127.0.0.1:5000/datetime').text
s.sendall(ts.encode('utf-8') + b'\0')
\end{lstlisting}

\subsection{Cambios en el servidor}

El servidor en C fue modificado para leer y registrar el timestamp recibido tras el opcode. Se añadió un nuevo argumento que se lee al inicio del handler:

\begin{lstlisting}[language=C,caption={Servidor: recepción del timestamp},label=lst:web3]
char op[MAX_NAME];
char ts[MAX_NAME];
char user[MAX_NAME];
read_string(client_sock, op);
read_string(client_sock, ts);
read_string(client_sock, user);
printf("s> OPERATION %s FROM %s AT %s\n", op, user, ts);
\end{lstlisting}

\subsection{Pruebas de validación}

Creamos un script \texttt{run_test.sh} que arranca el servicio web, lanza el servidor y realiza operaciones del cliente. This script checks that the timestamp is correct and that it appears in the log of the service:

\begin{lstlisting}[language=bash,caption={Script de prueba automatizada},label=lst:web4]
TS=$(curl -s http://127.0.0.1:5000/datetime)

...

grep -qE "^s> OPERATION REGISTER FROM liang AT [0-9]{2}/[0-9]{2}/[0-9]{4}" server.log
\end{lstlisting}

\subsection{Conclusión}

La parte 2 ha permitido mejorar el sistema añadiendo el tiempo a cada operación. Esta mejora facilita el análisis del comportamiento del sistema y pone las bases para el registro remoto con RPC en la parte 3.

\section{Parte 3: Sistema P2P con marcas temporales y concurrencia}

En esta tercera parte se amplía el sistema P2P añadiendo dos mejoras principales:

\begin{itemize}
\item Incorporación de \textbf{marcas temporales} en las operaciones mediante un servicio web externo.
\item Registro \textbf{concurrente de operaciones} en el servidor, asegurando consistencia.
\end{itemize}

\subsection{Servicio Web de Fecha y Hora}

Para registrar todas las operaciones en el servidor junto con su fecha y hora, se implementa un microservicio en Python usando \texttt{Flask}, que ofrece la fecha y hora actuales en formato \texttt{dd/mm/yyyy HH:MM:SS}:

\begin{lstlisting}[language=Python,caption={datetime_service.py},label=lst:datetime]
@app.route('/datetime')
def current_datetime():
ts = datetime.now().strftime('%d/%m/%Y %H:%M:%S')
return Response(ts, mimetype='text/plain')
\end{lstlisting}

El cliente realiza una petición HTTP al servicio antes de enviar cualquier operación al servidor central.

\subsection{Cliente con Timestamp}

El cliente Python se ha adaptado para:

\begin{itemize}
\item Obtener un timestamp del servicio web antes de cada operación.
\item Enviar dicho timestamp al servidor tras el nombre de la operación, ambos terminados en NUL (\texttt{\textbackslash0}).
\end{itemize}

Esto se encapsula en el método auxiliar \texttt{_send_op}. Ejemplo:

\begin{lstlisting}[language=Python,caption={Envío de operación con timestamp},label=lst:send_op]
@staticmethod
def _send_op(s, op_str):
s.sendall(op_str.encode('utf-8') + b'\0')
try:
ts = requests.get('http://127.0.0.1:5000/datetime').text
except Exception:
ts = ""
s.sendall(ts.encode('utf-8') + b'\0')
\end{lstlisting}

\subsection{Servidor Concurrente con Registro de Operaciones}

El servidor central se mantiene en C y se ha extendido para registrar cada operación en un archivo \texttt{server.log}. Cada entrada incluye:

\begin{itemize}
\item El tipo de operación.
\item El nombre del usuario.
\item El timestamp recibido desde el cliente.
\end{itemize}

Ejemplo de entrada de log:

\begin{verbatim}
s> OPERATION REGISTER FROM alice AT 10/05/2025 23:08:02
\end{verbatim}

Este log se genera usando \texttt{fprintf} dentro de secciones protegidas por un \texttt{mutex}, para garantizar la concurrencia segura entre hilos.

\subsection{Prueba de concurrencia}

El script \texttt{run_concurrencia_test.sh} lanza dos clientes en paralelo que realizan operaciones completas: \texttt{REGISTER}, \texttt{CONNECT}, \texttt{PUBLISH}, \texttt{LIST}, \texttt{GET FILE}, etc.

Este test verifica que:

\begin{itemize}
\item Todas las operaciones aparecen correctamente en \texttt{server.log}.
\item El fichero \texttt{rpc.log} recoge un resumen de las acciones.
\item Las transferencias P2P (	exttt{GET FILE}) funcionan correctamente incluso bajo concurrencia.
\end{itemize}

Fragmento de salida:

\begin{verbatim}
alice REGISTER 10/05/2025 23:08:02
bob REGISTER 10/05/2025 23:08:03
...
alice_copy.txt => Hola desde Bob
bob_copy.txt   => Hola desde Alice
+++ TEST DE CONCURRENCIA COMPLETADO +++
\end{verbatim}

\subsection{Makefile y scripts de prueba}

El Makefile se ha mantenido consistente, compilando el servidor y permitiendo limpieza rápida del entorno.

Los scripts \texttt{run_test.sh} y \texttt{run_concurrencia_test.sh} garantizan que el sistema funciona correctamente de forma secuencial y concurrente.

\subsection{Conclusión de la Parte 3}

Se ha implementado un sistema robusto que permite trazabilidad temporal y ejecución concurrente segura de operaciones. La integración con un servicio REST externo ilustra la interoperabilidad entre componentes distribuidos en diferentes lenguajes (Python y C), y prepara el sistema para futuras extensiones (auditoría, replicación, etc.).

\section{Batería de pruebas}

Con el objetivo de garantizar los casos extremos y el correcto funcionamiento de nuestro proyecto para todas las partes, se ha desarrollado una batería de pruebas automatizadas y manuales que cubren tanto el funcionamiento básico como situaciones límite. A continuación describimos esas pruebas.

\subsection{Pruebas funcionales (Parte 1)}

En la Parte 1 se diseñó el script \texttt{run\_test.sh}, que automatiza las siguientes pruebas:

\begin{itemize}
    \item Registro de usuarios y detección de duplicados.
    \item Conexión de usuarios y publicación de contenidos.
    \item Verificación de errores por contenido duplicado.
    \item Listado de usuarios y de ficheros publicados.
    \item Descarga de ficheros mediante conexión P2P.
    \item Borrado de contenidos y verificación del fallo controlado en GET\_FILE tras la eliminación.
\end{itemize}

Todas las pruebas son validadas mediante aserciones (\texttt{assert}) que comprueban tanto el contenido como el flujo esperado. El script imprime resultados tipo \texttt{[PASS]} y finaliza exitosamente sólo si todas las condiciones son correctas.

\subsection{Pruebas concurrentes (Parte 3)}

En la Parte 3 se desarrolló \texttt{run\_concurrencia\_test.sh}, que simula dos clientes en paralelo (Alice y Bob), ejecutando comandos de forma escalonada y concurrente. Se validan los siguientes aspectos:

\begin{itemize}
    \item Registro simultáneo de múltiples usuarios.
    \item Publicación de contenidos concurrente.
    \item Acceso a contenidos del otro usuario durante la sesión activa.
    \item Transferencia de ficheros entre clientes mediante P2P.
    \item Desconexión ordenada y acceso controlado.
\end{itemize}

Los resultados del test muestran en el log del servidor que las operaciones se intercalan temporalmente, lo que demuestra que el servidor es concurrente y no secuencial. Además, el contenido transferido con \texttt{GET\_FILE} se compara con el original usando \texttt{diff}, garantizando integridad.

\subsection{Casos límite y errores controlados}

Además de los casos correctos, se incluyen pruebas para validar el comportamiento ante:

\begin{itemize}
    \item Intento de listar usuarios sin estar conectado.
    \item Publicación de ficheros sin conexión previa.
    \item Descarga de ficheros no existentes.
    \item Solicitud de información de un usuario desconectado.
\end{itemize}

Estas pruebas permiten verificar que todos los errores previstos por el protocolo están correctamente codificados y gestionados tanto por el servidor como por los clientes.

\subsection{Conclusiones de las pruebas}

Las pruebas realizadas confirman:

\begin{itemize}
    \item La implementación del protocolo cumple con los requisitos.
    \item Las conexiones y transferencias P2P funcionan correctamente.
    \item El servidor maneja concurrencia de múltiples clientes.
    \item Todos los errores definidos se detectan y reportan de forma adecuada.
\end{itemize}

Este enfoque exhaustivo asegura la fiabilidad y estabilidad del sistema ante cualquier entrada esperada o errónea.


\clearpage

\end{document}

\clearpage

\end{document}
